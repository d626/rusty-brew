\section{Konklusjon}
\todo[inline]{Finn en fornugtig måte å konkludere noe fornuftig.}

Selv om prosjektet i nåværende form oppfyller kravene er det flere ting som kan forbedres. Som nevnt får brukeren veldig lite informasjon om hvorfor en opperasjon gik galt. En annen ulempe med systemets nåværende implementasjon er at systemet ofte vil krasje, heller en å returnere en feilmelding. De fleste tilfellene dette skjer er steder der en feil i utgangspunktet ikke skal kunne skje, som for eksempel at man ikke kan lese en fil som man vet finnes. Det er heller ikke sannsynlig at hele systemet krasjer, på grunn av bruken av tråder og seppareringen av funksjonalitet. Denne mangelen kan rettes på omtrent samme måte som den første, men vil føre til større endringer. Derimot vil ikke disse endringene gjøre at eventuelle klienter må endres. Det eneste klientene vil se er at de får bedre feilmeldinger, og at de får flere av dem, i stedet for at serveren krasjer. 

Ett annet område systemet kan forbedres på er hva som skjer hvis serveren stopper å virke. Fordi den er koblet til ett fysisk sytem burde det helst ha vært noen mekanismer inne for automatisk restart og lignende, men dette er ikke implementert. Dette kan forsvares med at man har nok kontrol over systemet til at man kan forsikre seg om at dette nærmest aldri vil skje, men man kan aldri være helt sikre. En måte å forbedre denne situasjonen på ville vært å lage en watchdog som kan restarte serveren, og at serveren har en måte å se om den regulerte en prosess når den krasjet, for i så fall å fortsette der den slapp. 

Til tross for disse ulempene fungerer systemet som det skal, og på grunn av designet er det nokså robust mot feil. Dette skyldes også delvis Rust sine innebygde mekanismer for å forhindre bugs. 